\documentclass[11pt]{article}

\usepackage{amsmath,amssymb}	% for mathematical notation
\usepackage{float} 				% put things exactly where I tell you!
\usepackage{multicol} 			% for column layout
\usepackage[utf8]{inputenc} 	% can we has UTF-8, plox

\title%
{%
	{\large Milestone Report}\\
	Compilers
}

\author%
{%
	Andreas Dall Løfgren\\
	{\tt jgc330@alumni.ku.dk}
	\and
	Paw Saabye\\
	{\tt hwx245@alumni.ku.dk}
	\and
	Casper B. Hansen\\
	{\tt fvx507@alumni.ku.dk}
}

\begin{document}

\clearpage
\maketitle
\thispagestyle{empty}
\begin{multicols}{2}
\begin{abstract}
This document describes the development of the Paladim programming language,
which is a pascal-like imperative language with functions and procedures, which
uses multidimensional regular arrays.

Each subtask will be addressed in its own subsection of the document.
\end{abstract}
\vfill
\columnbreak%
\tableofcontents
\end{multicols}

\clearpage
\section{Overview}
We were given a compiler code with a top-down parser. This is very inefficient 
and we had to change it to a bottom-up parser. We discarded all the settings 
called LL1 and replaced it with 'parser' which would refer to our own parser 
in attempt to create the context free Paladim language. 

\subsection{Planned schedule}
Having a development schedule helps us keep things organized and makes sure
that we can meet the deadlines. We provide our development schedule for our
own benefit and such that the instructor can provide feedback on possible
difficulties we may encounter, such that it can be  adjusted accordingly.
\begin{figure}[H]
	\centering
	\begin{tabular}{|l|c|c|c|}
		\hline
		{\bf Subtask} & {\bf Status} & {\bf Expected} & {\bf Deadline} \\ \hline
		Parser & N/A & N/A & 6th December 2013 \\ \hline
		Arithmetic and logic & N/A & N/A & 20th December 2013 \\ \hline
		Type inference- and checking & N/A & N/A & 20th December 2013 \\ \hline
		Code generation & N/A & N/A & 20th December 2013 \\ \hline
		Semantics & N/A & N/A & 20th December 2013 \\ \hline
	\end{tabular}
	\label{table:schedule}
	\caption{Our planned development schedule}
\end{figure}
The development schedule is, however, subject to change. For reasons given
above, and unforeseen difficulties may or may not force us to reschedule.

\section{Progress}
This section discusses the advancements, we have made in the above tasks.

\subsection{Parser}
Our initial attempt at writing the bottom-up parser in place of the top-down
LL1 parser was unsuccesful, and we had to revert all changes. This was because
of a lack of understanding the code and also as a consequence of some of us
being a bit rusty in the SML language. And so the bottom-up parser was
partially written while brushing up on the language in which it was written.

For the second attempt we simply went into the {\tt compile.sh} shell script,
and commented out the line which compiled the LL1 parser, making sure that any
code that relied on {\tt LL1Parser.sml} was exposed, and could be accounted
for by simply inspecting and migrating the dependency to the {\tt Parser.grm}
instead. This proved to be very advantageous as it also gave us a complete
overview of all the interdepencies and thereby allowed us to understand more
about what we were actually doing.

Firstly, we saw that {\tt Driver.sml} was performing calls on
{\tt LL1Parser.sml}, which was conveniently commented and prepared for the
transition over into {\tt Parser.grm}. So, commenting out and replacing these
lines with the appropriate ones, as directed in the assignment text, we turned
our attention to {\tt Lexer.lex}. The lexer depended heavily on
{\tt LL1Parser.sml} as all tokens were declared in that file, which gave rise
to declaring these in {\tt Parser.grm} instead, effectively removing that
dependency. When all tokens were replaced, the first rule was implemented in
{\tt Parser.grm}, the {\tt Prog} rule, which allowed us to see the parser
working and sparked the epiphany of how it actually functioned --- this was,
again, because we weren't used to working with the mosmlyac environment, and
so getting things started was highly based on pure experimentation and reading
the documentation, as well as examples provided as course materials.

As soon as one rule was implemented, most of the others followed right after.
The most challenging part of this subtask was understanding the structure of
the mosmlyac environment, as well as refreshing the SML language.

\subsubsection{Tests}
We didn't have enough resources to put extensive efforts into testing. We did,
however run each and every Paladim program provided in the DATA folder. By the
time the parser had been succesfully migrated to the bottom-up method using
context-free gramma in {\tt Parser.grm}, from the top-down {\tt LL1Parser.sml}
all of these programs ran without any parser errors.

\paragraph{Single-shot testing}
The testing has so far only involved single-shot testing, insofar as it has
been applicable.

Generally we have tested the implementation by recompiling the source, and
testing each file, thus, letting us see 1) the compiler output for any
warnings/errors and 2) the programme output, so that we may verify that it
is correct.

\subsection{Arithmetic and logic}
We have started working in Type.sml. Times, Div Or and Not should be 
at least partially funtional.

\subsubsection{Tests}
{\it not applicable at this time}%

\subsection{Type inference- and checking}
We have begun working on the type inference- and checking, however we 
have yet to gain sufficient overview over the task, such that we can 
produce any definite conclusions as to where we are.

\subsubsection{Tests}
{\it not applicable at this time}%

\subsection{Code generation}
We have not yet started on this part.

\subsubsection{Tests}%
{\it not applicable at this time}%

\subsection{Semantics}
We have not yet started on this part.

\subsubsection{Tests}
{\it not applicable at this time}%

\end{document}

