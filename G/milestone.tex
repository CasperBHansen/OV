\documentclass[11pt,a4paper]{article}

% Disagreed : a4paper document option gives the optimal number of chars per
% line (around 66 if I remember correctly).
\usepackage{a4wide} 			% save some rainforests
\usepackage{amsmath,amssymb}	% for mathematical notation
\usepackage{float} 				% put things exactly where I tell you!
\usepackage{multicol} 			% for column layout
\usepackage[utf8]{inputenc} 	% can we has UTF-8, plox

\title%
{%
	{\large Milestone Report}\\
	Compilers
}

\author%
{%
	Andreas Dall Løfgren\\
	{\tt jgc330@alumni.ku.dk}
	\and
	Paw Saabye\\
	{\tt hwx245@alumni.ku.dk}
	\and
	Casper B. Hansen\\
	{\tt fvx507@alumni.ku.dk}
}

\begin{document}

\clearpage
\maketitle
\thispagestyle{empty}
\begin{multicols}{2}
\begin{abstract}
This document describes the development of the Paladim programming language,
which is a pascal-like imperative language with functions and procedures, which
uses multidimensional regular arrays.

Each subtask will be addressed in its own subsection of the document.
\end{abstract}
\vfill
\columnbreak%
\tableofcontents
\end{multicols}

\clearpage
\section{Overview}
\ldots

\subsection{Planned schedule}
Having a development schedule helps us keep things organized and makes sure
that we can meet the deadlines. We provide our development schedule for our
own benefit and such that the instructor can provide feedback on possible
difficulties we may encounter, such that it can adjust it accordingly.
\begin{figure}[H]
	\centering
	\begin{tabular}{|l|c|c|c|}
		\hline
		{\bf Subtask} & {\bf Status} & {\bf Expected} & {\bf Deadline} \\ \hline
		Parser & N/A & N/A & 6th December 2014 \\ \hline
		Arithmetic and logic & N/A & N/A & N/A \\ \hline
		Type inference- and checking & N/A & N/A & N/A \\ \hline
		Code generation & N/A & N/A & N/A \\ \hline
		Semantics & N/A & N/A & N/A \\ \hline
	\end{tabular}
	\label{table:schedule}
	\caption{Our planned development schedule}
\end{figure}
The development schedule is, however, subject to change. For reasons given
above, and unforeseen difficulties may or may not force us to reschedule.

\section{Progress}
This section discusses the advancements, we have made in the above tasks.

\subsection{Parser}
\ldots

\subsubsection{Tests}
{\it not applicable at this time}%
\paragraph{One-shot testing}
The testing has so far only involved single-shot testing, insofar as it has
been applicable.

Generally we have tested the implementation by recompiling the source, and
testing each file, thus, letting us see 1) the compiler output for any
warnings/errors and 2) the programme output, so that we may verify that it
is correct.

\subsection{Arithmetic and logic}
\ldots

\subsubsection{Tests}
{\it not applicable at this time}%

\subsection{Type inference- and checking}
\ldots

\subsubsection{Tests}
{\it not applicable at this time}%

\subsection{Code generation}
\ldots

\subsubsection{Tests}%
{\it not applicable at this time}%

\subsection{Semantics}
\ldots

\subsubsection{Tests}
{\it not applicable at this time}%

\end{document}

