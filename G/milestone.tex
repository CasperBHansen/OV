\documentclass[11pt]{article}

\usepackage{amsmath,amssymb}	% for mathematical notation
\usepackage{float} 				% put things exactly where I tell you!
\usepackage{multicol} 			% for column layout
\usepackage[utf8]{inputenc} 	% can we has UTF-8, plox

\title%
{%
	{\large Milestone Report}\\
	Compilers
}

\author%
{%
	Andreas Dall Løfgren\\
	{\tt jgc330@alumni.ku.dk}
	\and
	Paw Saabye\\
	{\tt hwx245@alumni.ku.dk}
	\and
	Casper B. Hansen\\
	{\tt fvx507@alumni.ku.dk}
}

\begin{document}

\clearpage
\maketitle
\thispagestyle{empty}
\begin{multicols}{2}
\begin{abstract}
This document describes the development of the Paladim programming language,
which is a pascal-like imperative language with functions and procedures, which
uses multidimensional regular arrays.

Each subtask will be addressed in its own subsection of the document.
\end{abstract}
\vfill
\columnbreak%
\tableofcontents
\end{multicols}

\clearpage
\section{Overview}
We were given a compiler code with a top-down parser. This is very inefficient 
and we had to change it to a bottom-up parser. We discarded all the settings 
called LL1 and replaced it with 'parser' which would refer to our own parser. 

\subsection{Planned schedule}
Having a development schedule helps us keep things organized and makes sure
that we can meet the deadlines. We provide our development schedule for our
own benefit and such that the instructor can provide feedback on possible
difficulties we may encounter, such that it can be  adjusted accordingly.
\begin{figure}[H]
	\centering
	\begin{tabular}{|l|c|c|c|}
		\hline
		{\bf Subtask} & {\bf Status} & {\bf Expected} & {\bf Deadline} \\ \hline
<<<<<<< HEAD
		Parser & N/A & N/A & 6th December 2013 \\ \hline
		Arithmetic and logic & N/A & N/A & 20th December 2013 \\ \hline
		Type inference- and checking & N/A & N/A & 20th December 2013 \\ \hline
		Code generation & N/A & N/A & 20th December 2013 \\ \hline
		Semantics & N/A & N/A & 20th December 2013 \\ \hline
=======
		Parser & Done & On time & 6th December 2014 \\ \hline
		Arithmetic and logic & N/A & N/A & N/A \\ \hline
		Type inference- and checking & N/A & N/A & N/A \\ \hline
		Code generation & N/A & N/A & N/A \\ \hline
		Semantics & N/A & N/A & N/A \\ \hline
>>>>>>> 8f0025dc611c0c32343821f9756f38f845355864
	\end{tabular}
	\label{table:schedule}
	\caption{Our planned development schedule}
\end{figure}
The development schedule is, however, subject to change. For reasons given
above, and unforeseen difficulties may or may not force us to reschedule.

\section{Progress}
This section discusses the advancements, we have made in the above tasks.

\subsection{Parser}
The new parser was added to where LL1parser was before.
All of the dependencies in Lexer.lex and Driver.sml have been corrected and Parser.grm has been added instead.
The grammar rules specified in figure 3 in groupproject.pdf have been added.

During the project we had trouble with LVAL and value, which we solved by adding the position to the type
so that Absyn.LValue<LVAL> became Absyn.LValue<(LVAL*(int*int))> and Absyn.value similarly.

If-then was given precedence to if-then-else to resolve a conflict, where multiple statements were present.

\subsubsection{Tests}
We didn't have enough resources to put extensive efforts into testing. We did,
however run each and every Paladim program provided in the DATA folder. By the
time the parser had been succesfully migrated to the bottom-up method using
context-free gramma in {\tt Parser.grm}, from the top-down {\tt LL1Parser.sml}
all of these programs ran without any parser errors.

\paragraph{One-shot testing}
The testing has so far only involved single-shot testing, insofar as it has
been applicable.

Generally we have tested the implementation by recompiling the source, and
testing each file, thus, letting us see 1) the compiler output for any
warnings/errors and 2) the programme output, so that we may verify that it
is correct.

\subsection{Arithmetic and logic}
We have started working in Type.sml. Times, Div Or and Not should be at least partially funtional.

\subsubsection{Tests}
{\it not applicable at this time}%

\subsection{Type inference- and checking}
Same as Arithmetic and logic

\subsubsection{Tests}
{\it not applicable at this time}%

\subsection{Code generation}
We have not yet started on this part.

\subsubsection{Tests}%
{\it not applicable at this time}%

\subsection{Semantics}
We have not yet started on this part.

\subsubsection{Tests}
{\it not applicable at this time}%

\end{document}

