\documentclass[11pt,a4paper]{article}

%========== PACKAGES ==========%

\usepackage{a4wide} 			% save some rainforests
\usepackage{amsmath,amssymb}	% for mathematical notation
\usepackage[english]{babel}		% language definition
\usepackage{color}				% to make syntax pretty
\usepackage{enumerate}			% roman enumeration
\usepackage{fancyref}			% fancy references
\usepackage{float}				% for precise placement
\usepackage{graphicx}			% importing graphics
\usepackage{listings}			% code listings
\usepackage[utf8]{inputenc} 	% can we has UTF-8, plox
\usepackage{multicol} 			% for column layout
\usepackage{tikz} 				% pretty drawings

\usetikzlibrary{automata,arrows,positioning}	% make things easier

%========== DEFINITIONS ==========%

\definecolor{comment}{rgb}		{0.38, 0.62, 0.38}
\definecolor{keyword}{rgb}		{0.10, 0.10, 0.81}
\definecolor{identifier}{rgb}	{0.00, 0.00, 0.00}
\definecolor{string}{rgb}		{0.50, 0.50, 0.50}

\let\imp\to

%========== SETTINGS ==========%

\lstset
{
	% general settings
	numbers=left,
	frame=single,
	basicstyle=\footnotesize\ttfamily,
	tabsize=2,
	% syntax highlighting
	commentstyle=\color{comment},
	keywordstyle=\color{keyword},
	identifierstyle=\color{identifier},
	stringstyle=\color{string},
}

%========== DECLARATIONS ==========%

\title
{
	{\large Individual Assignment 2}\\
	Compilers
}

\author
{
	Casper B. Hansen\\
	Department of Computer Science\\
	University of Copenhagen\\
	{\tt fvx507@alumni.ku.dk}
}

%========== DOCUMENT ==========%

\begin{document}

\clearpage
\maketitle

\section{Writing Context-Free Grammars}
Write unambiguous grammars for the following languages over the alphabet
$\Sigma = \{a, b, c\}$:
% Write mosmlyacc grammar files for your grammars to check them. A grammar
% which does not cause conflicts is certain to be unambiguous. However, for
% (b), it will not be possible to get a grammar without conflicts. Why?

\subsection*{a \mdseries Words that match regular expression {\tt a*b*} which
contain more a's than b's.}
\begin{align*}
	S &\imp \epsilon | X \\
	X &\imp a|aXY \\
	Y &\imp \epsilon | aYb
\end{align*}

\subsection*{b \mdseries Palindromes (words that are identical when read
backward and forward).}
\begin{align*}
	S &\imp \epsilon | a | b | c | aSa | bSb | cSc
\end{align*}

\newpage
\section{LL(1)-Parser Construction}
Construct an LL(1) parser, taking the following grammar as a starting point:
\begin{align*}
	Z &\imp b | X Y Z \\
	Y &\imp \epsilon | c \\
	X &\imp Y | a
\end{align*}
with the terminal symbols $a$, $b$, and $c$.

\subsection*{a \mdseries Determine which nonterminals are nullable and
calculate first sets of all right-hand sides of the productions}
In determining which nonterminals are nullable we first describe the
production rules as a set of boolean equations
\begin{align*}
	Nullable(Z) &= Nullable(b) \lor Nullable(X) \land Nullable(Y) \land Nullable(Z) \\
	Nullable(Y) &= Nullable(\epsilon) \lor Nullable(c) \\
	Nullable(X) &= Nullable(Y) \lor Nullable(a) \\\\
	Nullable(\epsilon) &= {\tt true} \\
	Nullable(\alpha) &= {\tt false} \quad \text{where $\alpha \in \{a, b, c\}$}\\
	Nullable(Y) &= Nullable(Y) \\
	Nullable(XYZ) &= Nullable(X) \land Nullable(Y) \land Nullable(Z)
\end{align*}
From these equations we can now recursively iterate over the equations until
we find the calculations do not alter value. We do so by initially assuming
all to be {\tt false},
\begin{figure}[H]
	\center
	\begin{tabular}{c|cccc}
		{\bf Right-hand side} & {\bf Initial} & {\bf 1st} & {\bf 2nd} & {\bf 3rd} \\ \hline
		$\epsilon$ 	& {\tt false} & {\tt true} & {\tt true} & {\tt true} \\
		$a$ 		& {\tt false} & {\tt false} & {\tt false} & {\tt false} \\
		$b$ 		& {\tt false} & {\tt false} & {\tt false} & {\tt false} \\
		$c$ 		& {\tt false} & {\tt false} & {\tt false} & {\tt false} \\
		$Y$ 		& {\tt false} & {\tt false} & {\tt true} & {\tt true} \\
		$XYZ$ 		& {\tt false} & {\tt false} & {\tt false} & {\tt false} \\ \hline
		{\bf Nonterminal} & {\ } & {\ } \\ \hline
		$Z$ 		& {\tt false} & {\tt false} & {\tt false} & {\tt false} \\
		$Y$ 		& {\tt false} & {\tt true} & {\tt true} & {\tt true} \\
		$X$ 		& {\tt false} & {\tt false} & {\tt true} & {\tt true} \\
	\end{tabular}
	\label{table:nullable}
	\caption{Recursive iterations over the boolean equations}
\end{figure}
From the table above, we get that the nonterminals $X$ and $Y$ are nullable.
\newpage
We can now proceed to calculate the first sets of the right-hand sides in a
similar fashion, by writing up the recursive set equations.
\begin{align*}
	FIRST(Z) &= FIRST(b) \cup FIRST(XYZ) \\
	FIRST(Y) &= FIRST(\epsilon) \cup FIRST(c) \\
	FIRST(X) &= FIRST(Y) \cup FIRST(a) \\\\
	FIRST(\epsilon) &= \emptyset \\
	FIRST(\alpha) &= \{\alpha\} \quad \text{where $\alpha \in \{a, b, c\}$} \\
	FIRST(Y) &= FIRST(Y) \\
	FIRST(XYZ) &= FIRST(X) \cup FIRST(Y) \cup FIRST(Z)
\end{align*}
And then recursively iterate over the set equations until no further changes
are made.
\begin{figure}[H]
	\center
	\begin{tabular}{c|cccc}
		{\bf Right-hand side} & {\bf Initial} & {\bf 1st} & {\bf 2nd} & {\bf 3rd} \\ \hline
		$\epsilon$ 	& $\emptyset$ & $\emptyset$ & $\emptyset$ & $\emptyset$ \\
		$a$ 		& $\emptyset$ & $\{a\}$ & $\{a\}$ & $\{a\}$ \\
		$b$ 		& $\emptyset$ & $\{b\}$ & $\{b\}$ & $\{b\}$ \\
		$c$ 		& $\emptyset$ & $\{c\}$ & $\{c\}$ & $\{c\}$ \\
		$Y$ 		& $\emptyset$ & $\emptyset$ & $\{c\}$ & $\{c\}$ \\
		$XYZ$ 		& $\emptyset$ & $\emptyset$ & $\{a,b,c\}$ & $\{a,b,c\}$ \\ \hline
		{\bf Nonterminal} & {\ } & {\ } \\ \hline
		$Z$ 		& $\emptyset$ & $\{b\}$ & $\{a,b,c\}$ & $\{a,b,c\}$ \\
		$Y$ 		& $\emptyset$ & $\{c\}$ & $\{c\}$ & $\{c\}$ \\
		$X$ 		& $\emptyset$ & $\{a\}$ & $\{a,c\}$ & $\{a,c\}$ \\
	\end{tabular}
	\label{table:first-sets}
	\caption{Recursive iterations over the first sets}
\end{figure}
...
\begin{align*}
	Z &\imp b \\
	Z &\imp X Y Z \\
	Y &\imp \epsilon \\
	Y &\imp c \\
	X &\imp Y \\
	X &\imp a
\end{align*}
\begin{align*}
	T &\imp R \\
	T &\imp aTc \\
	R &\imp \epsilon \\
	R &\imp bR
\end{align*}

\subsection*{b \mdseries Calculate follow sets for all nonterminals (adding an
extra start production to recognise the end of the input, denoted by "\$")}
...

\subsection*{c \mdseries Determine the look-aheads sets of all productions and
put together a parse table for a predictive parser (as shown in the lecture
slides)}
...

\newpage
\section{SLR Parser Construction}
Make up a very small grammar which contains left-recursion, to demonstrate
that left-recursion is not a problem for LR-Parsing.
% If you want, you can use mosmlyacc for this task. When run with flag --output,
% mosmlyacc will produce a file with information about the underlying DFA for
% the generated SLR-parser. If you use mosmlyacc, hand in the .grm files you
% have used.

\subsection*{a \mdseries Show that your grammar does not generate conflicts
(by providing a parse table)}
...

\subsection*{b \mdseries Compare your grammar to an equivalent one that uses
right-recursion. How does the parse stack grow when parsing input?}
...

\end{document}
