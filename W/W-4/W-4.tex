\documentclass[11pt,a4paper]{article}

%========== PACKAGES ==========%

\usepackage{a4wide} 			% save some rainforests
\usepackage{amsmath,amssymb}	% for mathematical notation
\usepackage[english]{babel}		% language definition
\usepackage{color}				% to make syntax pretty
\usepackage{enumerate}			% roman enumeration
\usepackage{fancyref}			% fancy references
\usepackage{float}				% for precise placement
\usepackage{graphicx}			% importing graphics
\usepackage[utf8]{inputenc} 	% can we has UTF-8, plox
\usepackage{listings}			% code listings
\usepackage{multicol} 			% for column layout

%========== DEFINITIONS ==========%

\definecolor{comment}{rgb}		{0.38, 0.62, 0.38}
\definecolor{keyword}{rgb}		{0.10, 0.10, 0.81}
\definecolor{identifier}{rgb}	{0.00, 0.00, 0.00}
\definecolor{string}{rgb}		{0.50, 0.50, 0.50}

\let\imp\to

\newcommand{\specialcell}[2][l]{%
\begin{tabular}[#1]{@{}l@{}}#2\end{tabular}}

%========== SETTINGS ==========%

\lstset
{
	% general settings
	numbers=left,
	frame=single,
	basicstyle=\footnotesize\ttfamily,
	tabsize=2,
	% syntax highlighting
	commentstyle=\color{comment},
	keywordstyle=\color{keyword},
	identifierstyle=\color{identifier},
	stringstyle=\color{string},
}

%========== DECLARATIONS ==========%

\title
{
	{\large Individual Assignment 4}\\
	Compilers
}

\author
{
	Casper B. Hansen\\
	Department of Computer Science\\
	University of Copenhagen\\
	{\tt fvx507@alumni.ku.dk}
}

%========== DOCUMENT ==========%

\begin{document}

\clearpage
\maketitle

\section{Exercise 1}
% a) and b)  test your understanding of intermediate and machine-code
% generation. See class lectures "Intermediate -Code Generation" and
% "Machine Code", or Chapters 6 and 7  of "Introduction to Compiler
% Design" book.
%
% (do not worry  about the order in which fresh IL variables are generated,
% i.e., their numbers, as long as the translation is correct.)

\subsection*{a \mdseries Translate expression {\tt gcd(x+y, y+1) * 2} to
intermediate code. Place the result in {\tt t0} and assume that
{\tt vtable=[x->v, y->w], ftable=[gcd->\_GCD\_].}}
\begin{lstlisting}
t4 := v
t5 := w
t6 := 1
t7 := 2
t2 := t4 + t5
t3 := t5 + t6
t1 := CALL _GCD_(t2,t3)
t0 := t1 * t7
\end{lstlisting}

\newpage
\subsection*{b \mdseries Translate the following statement, i.e., {\tt Stat}
in grammar, to intermediate code, and then to MIPS code, where initially
{\tt vtable=[a->v, b->w]}.}
\begin{lstlisting}[language=C]
while (b != 0) and (a/b != 0) {
	if b < a then { a := a - b }
			 else { b := b - a }
}
\end{lstlisting}
\begin{lstlisting}
v0 := v
v1 := w
v2 := 0

	LABEL start
		t2 := v1
		t3 := v2
		t4 := t3 = v2
		t0 := not t4

		t5 := t2 / t3
		t6 := t5 = v2
		t1 := not t6
		
		IF t0 and t1 THEN GOTO then ELSE GOTO finish
		LABEL then
			t7 := t2
			t8 := t3
			IF t7 < t8 THEN GOTO if ELSE GOTO else
			LABEL if
				t9 := v0
				t10 := v1
				v0 := t9 - t10
			LABEL else
				t9 := v1
				t10 := v0
				v1 := t9 - t10
		GOTO start

	LABEL finish
\end{lstlisting}
\lstinputlisting{exercise_1b.asm}

\newpage
\subsection*{c \mdseries Make pattern/replacement pairs as in the "Machine
Code" lecture for each of the following intermediate-language instructions:\\
\begin{center}
	\begin{tabular}{|c|c|}
		\hline
		1 & {\tt rd := rs = rt} \\ \hline
		2 & {\tt rd := !r} \\ \hline
	\end{tabular}
\end{center}
where
\begin{enumerate}[i]
	\item = is an operator that returns 1 if its two arguments are equal and
	0 if they are not.
	\item ! is an operator that returns 1 if its argument is 0 and 0 if its
	argument is different than 0.
\end{enumerate}
Use the subset of MIPS supported by Mips.sml. Try to use as few instructions
and registers as you can in the translation.}
For the first (1) intermediate language instruction I have that
\begin{center}
	\begin{tabular}{|c|l|}
		\hline {\tt rd := rs = rt} &
		\specialcell{\tt li rd, 1 \\ \tt beq rs, rt, eq \\ \tt nop \\ \tt li rd, 0 \\ \tt eq:} \\ \hline
	\end{tabular}
\end{center}
And for the second (2) intermediate language instruction I get
\begin{center}
	\begin{tabular}{|c|c|}
	 	\hline {\tt rd := !r} & {\tt xori rd, r, 1} \\ \hline
	\end{tabular}
\end{center}
\newpage
\section{Exercise 2}
Assume we want to introduce a map higher-order array combinator in Paladim of
type:
\begin{center}
	{\tt map : (a -> b) * (Array r a) -> (Array r b)}
\end{center}
where $a$ and $b$ are basic types, i.e., one of {\tt int}, {\tt char} or
{\tt bool}, and $r$ is an arbitrary rank ({\tt int} variable). {\tt map}
semantics is that it applies the function argument to each basic-type element
of the (potentionally multi-dimensional) input array, and produces an array of
same rank $r$ (and dimension sizes) as the input array, but of basic-element
type $b$ i.e., an array of type {\tt (Array r b)}.

For example if
\begin{lstlisting}[language=C]
function f(y : int) : int
	return 2 * y;
\end{lstlisting}
and 
\begin{lstlisting}[language=C]
x : array of array of int;
x := { {1, 2, 3}, {4, 5, 6} };
\end{lstlisting}
then {\tt map(f, x)} results in
\begin{lstlisting}[language=C]
{ {2, 4, 6}, {8, 10, 12} }
\end{lstlisting}

Assuming that a function {\tt f : int->char} is available (known), and that a
variable {\tt x : array of array of int} is visible in the current scope;

\subsection*{a \mdseries Translate the call {\tt map(f, x)} in Paladim (using
while loops).}
\lstinputlisting{map.pal}

\subsection*{b \mdseries Implement the call {\tt map(f, x)} in Mips using the
array layout from G-Assignment and the most efficient translation you can
find. Assume vtable is {\tt [x -> regx]}
{\tt ftable} is {\tt [f -> "\_f\_"]} and {\tt HP} is the heap pointer. You may
also assume that you have a Mips instruction that calls a function, e.g.,
{\tt Mips.CALL(f, [re])}.}
% Hint: traverse the flat array representation)
\begin{lstlisting}
map:
	# stack push omitted

	la		$s0, _f_							# get the function address
	ori		$s1, $regx, 0					# get the array address
	ori		$s2, $zero, 2					# assume we know the rank (2)

	# read the meta data
	lw		$t0, 0($s1)						# get the first dimension
	lw		$t1, 4($s1)						# get the second dimension
	lw		$t2, 8($s1)						# get the stride
	mul		$s3, $t0, $t1					# calculate the array size

	# allocate array and store meta data
	ori		$v1, $HP, 0						# get the heap pointer for the new array
	addi	$HP, $HP, 16					# allocate header
	sw		$t0, 0($v0)						# store the first dimension
	sw		$t1, 4($v0)						# store the first dimension
	sw		$t2, 8($v0)						# store the stride
	sw		$HP, 12($v0)					# store the array pointer
	addi	$HP, $HP, 24					# allocate data

	# prepare to loop
	lw		$t0, 12($s1)					# get the array pointer
	la 		$t0, $t0
	lw 		$t1, 12($v0)					# get the new array pointer
	la 		$t1, $t1

	ori		$t2, $zero, 0					# counter, could break on address instead

	loop:
	lw		$a0, 0($t0)						# get the element
	CALL	$s0										# apply the function f, result expected in $v0
	sw		$v0, 0($t1)						# store the result

	addi	$t0, $t0, 4						# point to next element
	addi	$t1, $t1, 4						# point to next element

	beq		$t2, $s3, end_loop
	nop
	j loop
	end_loop:

	ori		$v0, $v1							# move return value into correct register
	
	# stack pop omitted
	
	jr $ra											# return (value stored in $v0)

\end{lstlisting}
The above code was written under the assumption that it need not be run, but
simply serves to show pseudo-wise how it might be evaluated as mips assembly.

\subsection*{c \mdseries Which version would be more efficient and Why?
(Between 2.b.) and the translation to machine code of the Paladim program in
2.a.)}
The most efficient of these implementations of map is the MIPS version, as it
traverses the flat array representation by pointer arithmetic, rather than
looking up the address of each individual entry of the array.

\end{document}
