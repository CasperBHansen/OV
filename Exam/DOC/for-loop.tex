%
% for-loop.tex
%
% Documentation of the for-loop implementation.
%

\section{For-loop}
Since no preparatory implementation has been made for the \code{for}-loop, in
order to implement the feature into the Paladim compiler we must; 1) introduce
the keywords, such that they can be recognized during the lexical analysis
stage, 2) add the token types and grammar, such that the statement can be
parsed correctly, 3) ensure that the expressions that form the inductive
variable and condition are of the proper type ---namely \code{int}-type--- in
the type-checking stage, and lastly 4) generate the necessary machine
instructions.

\subsection{Introducing the keywords}
The first thing we must do is to add the tokens to the parser in
\file{Parser.grm} (\appref{ch:appendix|sec:code|sub:forloop-tokens}), such
that they can be instantiated from the lexer.

Having a way to recognize the keywords \code{for}, \code{to} and \code{downto}
we can proceed to uncomment the abstract syntax tree definition in the files
\file{AbSyn.sml} and \file{TpAbSyn.sml} (both are identical,
\appref{ch:appendix|sec:code|sub:forloop-absyn}) of the \code{for}-loop, such
that it could be instantiated.

Now that we can instantiate the abstract syntax of a \code{for}-loop, we can
go on to define the grammar of it back in \file{Parser.grm}
(\appref{ch:appendix|sec:code|sub:forloop-grammar}). This, of course, also
requires us to define the \code{LoopDir} type
(\appref{ch:appendix|sec:code|sub:forloop-dirtype}), also in
\file{Parser.grm}, which concludes our interest in the parser.

Being able to parse the statement we can now safely introduce the lexical
analysis in \file{Lexer.lex} to these new keywords (lines 13--15), such that
we can allow for for parsing stage to get them
(\appref{ch:appendix|sec:code|sub:forloop-lex}).

\subsection{Type-checking}
Since all keywords are now recognized in the lexical analysis stage and
\code{for}-loop statements can be parsed correctly, we now have to type-check
such statements (\appref{ch:appendix|sec:code|sub:forloop-type}). Since the
inductive variable is used by the \code{for}-loop as a counter, I would reason
that the initialization and condition expressions need to be of type
\code{int}, as indicated to the type-checker on lines 6--7, for which the
type of the resulting typed expressions is retrieved on lines 9--10, and
verified that this is indeed the case on line 11. If they are both of type
\code{int} we can return the typed \code{for}-loop statement, otherwise we
raise an exception.

\newpage
\subsection{Machine-code generation}
The type-checked \code{for}-loop statements are now ready to be generated into
machine code. In \file{Compiler.sml} we add a new case to the
\code{compileStmt}. I suspect the specific case where its parameters include
the \code{exitLabel} is in preparation of introducing\footnote{This is
{\it not} part of my task set.} the keywords \code{break} and \code{continue},
then I reason that it should be included in the argument list, instead of
leaving a wildcard, such that we can reference that variable at a later time
if needed.

\paragraph{Compiled assembly structure}
To produce the necessary machine code
(\appref{ch:appendix|sec:code|sub:forloop-machinecode}), as done in
\file{Compiler.sml}, is relatively straightforward. The only quirky detail
worth elaborating on is how we will handle the loop condition; I reason that
going from $a$ to $b$ means that we need to stop iterating at {\it or} after
going past $b$, regardless of the direction.
\begin{enumerate}
	\item Retrieve the register in which the inductive variable exists (line
	2)
	\item Compile the initialization and condition expressions into their
	respective register locations (\code{loc} and \code{check}, lines 7--8),
	as well as the statement block.
	\item If the direction of the \code{for}-loop is positve (e.g. \code{for
	\dots to}) then increase the inductive variable by one after each pass
	(line 10), otherwise (e.g. \code{for \dots downto}) then decrease the
	inductive variable by one after each pass (line 11)
	\item Place the assembly in order; compilation of initializations (line
	13) and the entry loop label
	\item If the direction of the \code{for}-loop is positive (e.g. \code{for
	\dots to}) then set the flag register \code{lflag} if the condition is
	less than the inductive variable (line 16), otherwise negative (e.g.
	\code{for \dots to}) then set the flag register \code{lflag} if the
	inductive variables is less than the condition (line 17)
	\item Branch out of the loop if the flag is {\it not} zero\footnote{Note
	that this is compared to a register which contains the value zero at all
	times} (line 18)
	\item Place the statement block code immediately followed by the direction
	code, and lastly the loop jump instruction and exit label (lines 19--20)
\end{enumerate}

\subsection{Evaluation}
In a program (\appref{ch:appendix|sec:testcases|sub:for-assert}) based
largely on the example given in the task assignment text, in lack of
imagination, a number is asked of
with which it produces an array of integers. The array is filled in, using a
\code{for-to} statement with values $i + 1$. In the same manner, only this
time using a \code{for-downto} statement the array values are decremented
by 1.

The invariant maintained is that at the end $a[i] = i,{\ }\forall i < n$,
where $n$ is the dimension of the array $a$. A test procedure then performs an assertion
test on each index into $a$ by \code{AND}'ing the expression for all $i < n$.
If the result of this conjunction is \code{true} then all fulfill the
invariant, if not, however, the assertion fails. In any case the result is
output to the console as either {\it Passed!} or {\it Failed!}.

For reference, the results of this test
(\appref{ch:appendix|sec:tests|sub:forloop}) has been supplied in the appendix.
