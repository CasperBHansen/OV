%
% and-op.tex
%
% The documentation of the AND-operation extension.
%

\section{Bitwise AND operation}
In the implementation handed out, the type-checker (\file{Type.sml}) checks
that the operands of an \code{AND}-operation are of type \code{bool}. We would
like to extend this to accommodate the types \code{int} and \code{char}, for
which such expressions the evaluation must be bitwise.

\subsection{Type-checking}
First, we must allow the expression operands to be polymorphic, while still
maintaining that both operands should be of the same basic type. We do this by
altering the \code{typeCheckExp} function in \file{Type.sml}
(\appref{ch:appendix|sec:code|sub:and-type}). To reflect that the
function should accept operands of any basic type, as long as they are of the
same type, we would no longer infer the operand expressions to be of any
particular type, hence we passed an expected type of \code{UnknownType} on
lines 2--3. Having done so, we simply need to check whether or not the
operands are of the same basic type, and we do this one line 5, after having
retrieved their respective types on line 4. If they are of the same basic
type we return the typed expression on line 6, otherwise we raise an error on
lines 7--8. 

\subsection{Machine-code generation}
Now that the type-checker will allow polymorphic types, we can turn our
attention to the machine-code generation --- the code changes described occur
in \file{Compiler.sml} (\appref{ch:appendix|sec:code|sub:and-machinecode}). I
reason my implementation of the \code{AND}-operator by recognizing that
\code{AND}'ed expressions of operand-type \code{bool} type must remain as is.
However, since both \code{int} and \code{char} should be handled in the same
manner, in any other case we can assume that this is a bitwise
\code{AND}-operation.

By this reasoning we arrive at the conclusion that we need only check, as
done on line 7, whether or not the operands are of type \code{bool}. If that
is the case then we need only do what was already done, now on lines 8-11. In
any other case (hence the wildcard case) we can assume that we should handle
this case as a bitwise operation, as done on line 12. In both cases we do, of
course still need to compile (lines 2--5), as was already done before, and
prepend the code (line 12) of the operands.

\subsection{Precedence}
I would argue, that bitwise operations should have higher precedence than any
logical (boolean) operation. For instance the expression \code{a and b = c and
d} should evaluate as \code{(a and b) == (c and d)}, for the simple reason
that more often than not you would like this behavior --- especially when
working with bitmasks, where you would \code{AND} or \code{OR} together a
series of enumerations and then compare the result with another mask.

Now, I did state {\it any} bitwise operation, and I did state {\it any}
logical operation, therefore there needs to be a distinction between these
operators, as there is in languages like C++, Java, etc. So the implications
for the implementation must introduce new operators specific such that such a
distinction can be made by the parser, and hence compiler. C++ for instance
uses \code{\&\&} for logical \code{AND}, and \code{\&} for bitwise.

\subsection{Evaluation}
In evaluating the extension of the \code{AND}-operator, I have carried out a
few simple assertion tests (\appref{ch:appendix|sec:testcases|sub:and-assert})
The tests are constructed by writing a procedure for each datatype, taking
three arguments; a left-hand side, right-hand side and an expected value. The
procedure carries out the \code{AND}-operation on the passed left-hand side
and right-hand side and compares the result to the expected result value. If
the result is equal to the expected value it writes ``Passed!'' to the
console, and if not it writes ``Failed''. For the results of this test program
\appref{ch:appendix|sec:tests|sub:and-assert}.

A separate Paladim program has been supplied for evaluating expressions
interactively --- compile and run the \file{DATA/and.pal} program to try it
at your discretion with your own data to verify.

