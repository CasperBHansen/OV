%
% map.tex
%
% Documentation of the implementation of the higher-order map function.
%

\section{Map function}
\label{sec:map}
The \code{map} function has been prepared for implementation up until the
type-checking stage of the compiler. In order to implement the \code{map}
function we will need to check for a valid argument list, that the number and
types of arguments matches according to the type rule
(\txtref[Type Rule]{sec:map|sub:typecheck}) of the \code{map} function, since it is
polymorphic.

\subsection{Type-checking}
\label{sec:map|sub:typecheck}
The \code{map} function is polymorphic, therefore we need to establish a rule
of type-checking calls to this function.

\paragraph{Type rule}
\begin{enumerate}
	\item Exactly 2 arguments must be supplied
	\item The first parameter must be a function \code{f}
	\begin{enumerate}
		\item Function \code{f} must take exactly one argument of type $\alpha$
		\item Function \code{f} must return exactly one value of type $\beta$
	\end{enumerate}
	\item The second parameter must be an array \code{A} of basic type
	$\alpha$
\end{enumerate}

In \file{Type.sml} (\appref{ch:appendix|sec:code|sub:map-type}) we ensure that
when the \code{map} function gets called it will raise an exception if the
argument list is not valid (e.g. the length is not exactly 2) by adding a case
with exactly two arguments on line 1, and leaving the catch-all pattern (of
the argument list) to raise the exception on lines 19--21. Expanding on that,
exploiting the pattern-matching feature of SML in our favor we can further
narrow the argument list of the accepting case down to only identifier
expressions for the first argument and pulling out the corresponding
identifier, again on line 1. This effectively takes care of the first and
part of the second type rule clause, as defined above.

Having the identifier of the supplied arguments available we can now look it
up in the symbol table on lines 2--5, giving us the function argument and
return types --- if the identifier is not found in the function table we raise
an error, which completes the second type rule. We then type-check the second
argument on line 7, and make sure it is an array on lines 9--11, pulling out
the rank and basic type on line 8, producing an error if it is not an array,
which takes care of part of the third type rule.

Finally, knowing the types of the function and array arguments, we can check
that 1) the function takes exactly one argument, and 2) the basic types of the
function and array arguments are the same, according to type rules 2(a) and 3,
all on line 13. If this is the case, we then return the typed \code{map}
function call\footnote{The reason it is declared as a variable on line 14 is
to do a correctness-check, as commented on line 15} on line 16, as defined in
\file{TpAbSyn.sml}, otherwise raise an exception.

\newpage
\subsection{Machine code}
\label{sec:map|sub:machinecode}
As far as I can see from other parts of the code, I took a different approach
as to how to traverse the arrays (I use three separate loops). This is
probably a symptom of being used to high-level programming, as opposed to
assembly programming. Nevertheless, it works as expected, and the only
side-effect is that the program will take up slightly more space on disk, this
doesn't make the implementation inefficient in terms of execution performance
however.

First, I will state my assumptions and hence reasoning how I implemented
(\appref{ch:appendix|sec:code|sub:map-machine}) the compiled instructions for
the \code{map} function.

\paragraph{Assumptions}
\begin{enumerate}
	\item For an array $\alpha$ of rank $r$
	\begin{enumerate}
		\item The first $r \cdot 4$ bytes contains the dimensions of the array
		$\alpha$.
		\item The following $(r - 1) \cdot 4$ bytes contains the strides of
		the array $\alpha$.
		\item The following, and last $4$ bytes contains the pointer to the
		actual array $\alpha$.
	\end{enumerate}
	\item It follows by simple arithmetic reduction
	(\appref{ch:appendix|sec:equations|sub:array-head-size}) from (1) that an
	array of
	rank $r$ will have a header of size $h_{size} = 8 \cdot r$. And that
	\begin{enumerate}
		\item The bytes containing the dimensions are stored in the first
		$\frac{h_{size}}{2}$ bytes.
		\item The bytes containing the strides and array pointer are stored in
		the following $\frac{h_{size}}{2}$ bytes.
	\end{enumerate}
\end{enumerate}

Based on these assumptions, I will explain the compiled structure of the
corresponding assembly instructions on the following page.

\newpage
\paragraph{Compiled assembly structure}
\begin{enumerate}
	\item Initialize registers; compile source array into register \code{src}
	(line 13), copy the heap pointer into the result register \code{place}
	(line 33), copy the address into the \code{dst} register (line 34) and
	initialize the \code{size} register to a value of 1 (line 35)
	\item Allocate half of the header (line 36) and loop through the bytes
	containing the dimensions (lines 37--45)
	\begin{enumerate}
		\item Load the value at the address of \code{src} into the temporary
		register \code{tmp} (line 39), and store at the address of \code{dst}
		(line 40)
		\item Multiply into \code{size} the current value at that register
		with the value stored in \code{tmp} (line 41)
	\end{enumerate}
	\item Calculate the exact byte-aligned size of the array (lines 46--51)
	\item Allocate the remainder of the header (line 52) and loop through the
	bytes containing the strides and pointer, copying each one in \code{src}
	into the corresponding address of \code{dst} (lines 53--60)
	\item Since \code{dst} now holds the address of the source array, we need
	to store the correct address, so we store the heap pointer into the
	address of \code{dst}, but with an offset of $-4$ bytes, because the loop
	took us exactly $4$ bytes past the correct address (line 61)
	\item Allocate the array data segment by adding the contents of the
	\code{size} register to the heap pointer, and loop through it (lines
	63--71)
	\begin{enumerate}
		\item If the source is of type \code{int} load a word from \code{src}
		into the temporary register \code{tmp}, otherwise load a byte (lines
		65, calculated on lines 22--25)
		\item Call the argument function on \code{tmp} and store its result
		back into the same register (line 66)
		\item If the destination is of type \code{int} store a wod from
		\code{tmp} into the destination register \code{dst} (line 67,
		calculated on lines 27--31)
		\item Advance the address pointers \code{src} and \code{dst} by the
		appropriate number of bytes (lines 68--69, calculated on lines 22--31)
	\end{enumerate}
\end{enumerate}

\newpage
\subsection{Efficiency}
The most prominent efficiency topic to be discussed is probably the fact that
the instructions does only what it needs to do, for the most part, although
quite I'm sure that there are still some optimizations that could make it even
faster.

The reason I implemented it as three distinct loops is that it allows for
intermediate computations while it loops. The first loop does the array size
calculation, with a minor follow-up afterward to account for byte-alignment.
The second does not depend on anything other than the header size, so it can
simply do a direct copy, and with a minor follow-up to correct the array
pointer immediately after. The last loop is highly dependent on previous
calculations at run-time, and thus isolating this in its own loop allows for
the flexibility needed.

No code generated is obsolete either, since all dependencies have been
accounted for and put into variables determined by these dependencies
beforehand. This ensures little or no code bloating.

\subsection{Evaluation}
Testing of the implementation was done by assertion tests; that is, I wrote a
Paladim program (see \appref{ch:appendix|sec:testcases|sub:map-assert}) which
defines functions \code{f : $\alpha \imp \beta$} and \code{g : $\beta \imp
\gamma$}. I then supplied a call to \code{map} with function \code{f} and an
array containing elements of basic type $\alpha$, the result of which is piped
through to a call to \code{map} with function \code{g}, the result of which is
asserted to be some value of type $\gamma$. Each assertion outputs the result
of the assertion (i.e. whether or not the assertion succeeded).

